\chapter{Discussion}
%1. Begin with a restatement of your research question, followed by a statement about whether or not, and how much, your findings "answer" the question.  These should be the first two pieces of information the reader encounters.
%ci sono delle mutazioni deleterie nel DNA mitocondriale o nei geni espressi nel mitocondrio?
The genetic causes of miscarriages are not yet fully known, there are areas of the genome that are still completely unexplored and  mtDNA is an important element in this kind of study, in fact,through life mutations can occur in one mtDNA and its descendants creating genetic variability of mitochondrial DNA, some of these mutations might be deleterious. So, the analysis of the mtDNA sequences and the nuclear genes (involved in mitochondrial processes) of embryos from pregnancy loss can informs on genes putatively causative of diseases \\

%1a breve ricapitolazione dei risultati 
 With this work I demonstrated that it is possible to identify variants in mitochondrial DNA to cause diseases using high-coverage whole-genome sequence data.
 I analyzed the mtDNA sequences of the 10 embryos from mothers that experienced pregnancy loss to discover 276 unique variable sites in all samples. Per each sample, the number of variable sites vary between 22 and 103. Using a filtering procedure  based on genomic annotations I identified 4 variants that are likely to be deleterious,These variants are found in the \textit {MT-ND2}, \textit{MT-ATP6} and \textit{MT-ND6} genes. 
Pathogenic variants of the mitochondrial gene \textit{MT-ND2} are known to cause mtDNA-associated Leigh syndrome, as are variants of \textit{MT-ATP6} and \textit{MT-ND6}. 
Furthermore I have analyzed variation in nuclear genes whose product are involved in mitochondrial processes  
I further report on three variants (rs2159132-rs10935321-rs16872235) either because they are shared among samples and because they are the most deleterious according to vep. These variants are found in the \textit{COX10}, and \textit{MRPS22} \textit{GFM2} genes.
These avariants are associated with elements of the respiratory chain and mutations in these components are associated with pathologies such as leigh syndrome.
Further analysis concerns the determination of the haplogroup to 
define the geographic distribuitiom of phylogenetic clusters among
samples. My analysis reveals that six out of ten sequences belong to
haplogroup H, typical of West Eurasia.  


%2. Relate your findings to the issues you raised in the introduction. Note similarities, differences, common or different trends.  Show how your study either corraborates, extends, refines, or conflicts with previous findings.%3. If you have unexpected findings, try to interpret them in terms of method, interpretation, even a restructured hypothesis; in extreme cases, you may have to rewrite your introduction. Be honest about the limitations of your study.
% rispetto a quanto detto nel paragrafo dell'introduzione mt DNA e diseases (e sopratutto in miscarriages) che cosa aggiungi/confermi/discrediti

Compared to what already available in literature most of the
diseases are related to the respiratory chain and the processes that
involve it. The analysis I carried out confirm what I found in my
studies , in fact, all the variants that I found are involved in
that processes. 
In addition, these variants that I found are all associated with the same mitochondrial disorders.
My findings about Haplogroups are concordant with the demographic data that were avaialble for these samples.  

%4. State the major conclusions from your study and present the theoretical and practical implications of your study.




%5. Discuss the implications of your study for future research and be specific about the next logical steps for future researchers.

Overall, I demonstrated that the analysis of mitochondrial DNA with whole-genome sequencing  can help to clarify part of the causes of pregnancy loss and provides essential indications for the realization of a larger study.


